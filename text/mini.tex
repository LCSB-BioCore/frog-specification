
\chapter{miniFROG report format}
\label{chap:mini}

While the FROG analysis is helpful for new models, the task of retrospective curation of already published models demands comparison of model performance with results reported in the manuscript.
This comparison should ensure that the right model version was shared, and that the model reproduces the published results.

To facilitate the retrospective curation, `miniFROG' report serves as a template to specify key observations from the manuscript, and declare their agreement with a FROG report.
That serves as a check of the reproducibility of the published data, with respect to the published model version.

\section{miniFROG report fields}

miniFROG report consists of a spreadsheet (or other accessible tabular data format) that must contain a set of entries that consist of the following labeled fields:
\begin{itemize}
\item \texttt{Publication Information} --- identifier of the publication, preferably a full citation accompanied with a DOI.
\item \texttt{Organism Name} --- name of the organism, provided for reference.
\item \texttt{Model Repository} --- name of the repository where the model is deposited, such as \texttt{BioModels}.
\item \texttt{Model Identifier} --- deposition ID of the model, such as \texttt{BIOMD00000123}. For use in BioModels repository, the fields \texttt{Model Repository} and \texttt{Model Identifier} may be merged into a single field \texttt{BioModel Identifier}.
\item \texttt{Simulation constraints} --- free text description of the additional constraints placed on the model, preferably with a reference to publication section that describes the constraints.
\item \texttt{Gene/reaction involved in the constraint} --- identifier of the constrained object, if applicable.
\item \texttt{Type of constraints in the simulation} --- free text description of the applied constraint; typically describing either a gene or reaction knockout, or a partial constraint to a fixed numeric bound.
\item \texttt{Tools used for simulation} --- free text description of software tools used to run the simulation.
\item \texttt{Results from publication} --- numeric output(s) obtained from the simulation and reported in the manuscript, preferably accompanied to a reference to exact publication section where the reesult is reported.
\item \texttt{Results predicted from FROG} --- numeric values in a FROG analysis that should correspond to the simulation result, or is easily convertible to the simulation result using a simple formula or algorithm (use `remarks' field to describe the conversion process).
\item \texttt{Type of F/R/O/G analysis} --- specify which part of FROG report contains the result predicted by FROG, preferably as a file name that contains the numeric value.
\item \texttt{Line in FROG Report} --- specify what line of the filename specified by FROG analysis type contains the result predicted by FROG.
\item \texttt{Validation of the simulation (Yes/No)} --- use a single work `Yes' or `No' to mark if FROG analysis was able to validate the simulation result. Note that failure to validate does not imply result invalidity, it merely flags a reproducibility problem in some of the many parts of the pipeline, including the FROG analysis software.
\item \texttt{Remarks} --- include any extra information or references necessary to validate the analysis.
\end{itemize}

miniFROG report should be prepared manually using this template, and submitted as an additional file during model submission to a model repository, or attached to an existing model deposition in a repository.

\section{miniFROG interpretation and examples}

Importantly, miniFROG reports are supposed to be interpreted and evaluated by human curators.
That is a required condition for the post-submission curation of models and publications that were created before FROG methodology was developed, and implies the human-accessible formating of most of the fields.
As the main constraints, the fields must describe a clear way to find the corresponding results in a FROG report and in the publication, and must obviate a straightforward algorithm to compare the numeric values that can be followed by the curator.
Conversely, software authors must not assume machine readability of the miniFROG results.

As an example, the model BIOMD0000001046\footnote{\url{https://www.ebi.ac.uk/biomodels/BIOMD0000001046}} by \textcite{raman2005flux} is curated using FROG test suite and miniFROG standards.
Additional examples of miniFROG reports are currently available online as spreadsheets\footnote{\url{https://docs.google.com/spreadsheets/d/1q4rre0guWh9Q2AYeBzfWHT-5cKBkMToc/edit\#gid=859592364}}.\todo{save this in the repository}
