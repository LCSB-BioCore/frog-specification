\chapter{FROG report format}
\label{chap:spec}

A FROG report contains information about one model present in one file.
The report consists of several files that must be saved in a single directory structure, such as in a single folder or in a single sub-directory of an archive file (see \cref{sec:archives}).
Independently of the storage method, the report files should be extractable into a single filesystem-level directory to allow easy data access for various tools.

\section{Archive files with FROG reports}
\label{sec:archives}

FROG reports may be stored and distributed in archive files. The formats of archive file should be preferably ZIP, with file extension \texttt{.zip}. The archive file must contain the report files in the top-level directory (i.e., no sub-directories should be used).

For model publishing purposes, the files should be stored in the OMEX (Open Modeling EXchange) format together with the model, as specified by \citeauthor{bergmann2014omex}, and may optionally be accompanied by an annotation file that follows the OMEX metadata recommendations by \citeauthor{Neal2018}.
The OMEX format identifiers for individual FROG files are listed in~\cref{sec:files} (\cref{tab:omex}).

\section{Directory layout}
\label{sec:files}

The required files that must be present in a FROG report are listed in \cref{tab:files}.
The formatting and contents of the files are described in the following sections (as referenced in the table). Additionally, the file format is specified by OMEX via \texttt{identifiers.org}; the current format versions are listed in \cref{tab:omex}.

\begin{table}\tablefont
\begin{tabular}{lp{.3\linewidth}ll}
\toprule
File name & Contents & Format & Description \\
\midrule
\texttt{metadata.json}
 & Basic information about the model file, information about the FROG report software.
 & JSON (\cref{sec:json})
 & \cref{sec:meta}
 \\\addlinespace
\texttt{01\_objective.tsv}
 & Expected optimal objective values.
 & TSV (\cref{sec:tsv})
 & \cref{sec:obj}
 \\\addlinespace
\texttt{02\_fva.tsv}
 & Expected reaction flux variability at the model optima.
 & TSV (\cref{sec:tsv})
 & \cref{sec:fva}
 \\\addlinespace
\texttt{03\_gene\_deletion.tsv}
 & Expected behavior of the model under single-gene deletions.
 & TSV (\cref{sec:tsv})
 & \cref{sec:genes}
 \\\addlinespace
\texttt{04\_reaction\_deletion.tsv}
 & Expected behavior of the model under single-reaction deletions.
 & TSV (\cref{sec:tsv})
 & \cref{sec:rxns} \\
\bottomrule
\end{tabular}
\caption{Files required in a FROG report.}
\label{tab:files}
\end{table}

\begin{table}\tablefont
\begin{tabular}{lp{.6\linewidth}}
\toprule
File name & OMEX identifier \\
\midrule
\texttt{metadata.json}
 & \url{https://identifiers.org/combine.specifications:frog-metadata-version-1}
 \\
\texttt{01\_objective.tsv}
 & \url{https://identifiers.org/combine.specifications:frog-objective-version-1}
 \\
\texttt{02\_fva.tsv}
 & \url{https://identifiers.org/combine.specifications:frog-fva-version-1}
 \\
\texttt{03\_gene\_deletion.tsv}
 & \url{https://identifiers.org/combine.specifications:frog-genedeletion-version-1}
 \\
\texttt{04\_reaction\_deletion.tsv}
 & \url{https://identifiers.org/combine.specifications:frog-reactiondeletion-version-1}
 \\
\bottomrule
\end{tabular}
\caption{OMEX format URLs for files archived in a FROG report. The versions in URLs may differ based on the current status of the specification.}
\label{tab:omex}
\end{table}

\section{Report file formats}

This section describes the encoding of information into file contents used by FROG reports.

\subsection{JSON}
\label{sec:json}

JSON-formatted FROG report files must adhere to the JSON formatting specified by RFC~8259~\cite{rfc8259}.
The character encoding of the JSON-formatted file must be UTF-8 (see \cite[][section 8.1]{rfc8259}).

Individual files may require adherence to a specified JSON Schema~\cite{pezoa2016jsonschema}, which shall be provided in JSON format in the respective section that describes the files' contents.

\subsection{TSV}
\label{sec:tsv}

TSV stands for Tab-Separated Values file. The format is defined as a matrix of text fields that do not contain characters for a tab and a newline (character values 9 and 10, respectively, commonly written as \verb|\t| and \verb|\n|), where the tab is used to separate individual per-column values in each row, and the newline is used to separate individual rows.

All fields in a FROG TSV file must be encoded in UTF-8 and must not contain tab or newline characters.
The first row of each FROG TSV file (a \emph{header row}) must contain column names as prescribed in the appropriate specification section. The columns must be present in the order prescribed by the specification.
All other rows except for the header row are called \emph{entry rows}.

IANA describes the format as media type `tab-separated-values'~\cite{ianatsv}.

For example, the TSV file in~\cref{lst:ecoli-obj} contains one header row and one entry row, and the entry row contains (in order) 2 plain text values, one objective status value, and one numeric value.
For clarity, the listing highlights the tab characters with light gray.

\subsubsection{Numeric and special values in TSV files}
\emph{Numeric} fields must be converted to decimal representation. The decimal representation may include a leading minus sign, decimal point, and an integer exponent separated by \verb|e| (for example, valid encodings of $-1234.5\times10^2$ include \verb|-123450|, \verb|-12.345e4|, and \verb|-12345000e-2|). Rounding of non-representable and infinite numbers may be performed, but it should preserve the outcome of the report comparison, as described in~\cref{sec:cmpdata}.

Missing numeric values must be represented by an empty field. Missing values typically occur when a value is expected in a field (such as in the place of the optimal value of the model), but it can not be computed for some reason (such as model infeasibility).

\emph{Solver status} fields contain a special value that describes the outcome of an optimization algorithm. FROG currently describes only two kinds of statuses:
\begin{itemize}
\item a feasible solution was found, and the returned result is optimal, represented by the keyword \verb|optimal|,
\item a feasible solution was not found, represented by the keyword \verb|infeasible|.
\end{itemize}
Solver status fields must contain exactly one of the above keywords.

\section{Report file contents}

\subsection{Metadata file}
\label{sec:meta}

The metadata file's contents are formatted as JSON containing a single JSON object.
The primary purposes of the file include the following:
\begin{itemize}
\item identifying the model file being described,
\item describing the software that was used to generate the report.
\end{itemize}

The expected properties and sub-properties of the JSON object are listed in \cref{tab:metadata}. The fields labeled `required' must be present in a FROG report. The fields labeled as `optional' should be present, and the software should implement generation and testing of these fields, but the report is valid even with the models missing.

The JSON Schema that may be used to validate the JSON metadata format is available from the specification repository\footnote{\url{\frogrepobase/schema/frog-metadata-v1.json}}.

\begin{table}\tablefont
\begin{tabular}{llp{30em}}
\toprule
Property & Presence & Value description \\
\midrule
\verb|frog_version|
 & required
 & 3-number tagged version string of the FROG standard that is followed by the report, such as \verb|0.0.2|
 \\
\verb|frog_id|
 & optional
 & identifier of the FROG report that is unique within a given data repository
 \\
\verb|frog_date|
 & optional
 & date of creating the FROG report, preferably in ISO-8601~\cite{iso8601} format, for example \texttt{2023-02-23}
 \\
\verb|model_filename|
 & required
 & expected base name of the model file, such as \verb|e_coli_core.xml|
 \\
\verb|model_md5|
 & required
 & 32-hexadecimal-digit MD5 hash~\cite{rfc1321} of the model file contents
 \\
\verb|model_sha256|
 & optional
 & 64-hexadecimal-digit SHA256 hash~\cite{national2008secure} of the model file contents
 \\
\verb|model_file_url|
 & optional
 & URL where the model file can be obtained, preferably pointing to a long-term archival repository
 \\
\verb|model_development_url|
 & optional
 & URL of a website or public repository where the model is developed, preferably offering a chance to contact the model authors or submit corrections of the model
 \\
 \verb|model_reproducibility_url|
  & optional
  & URL of a website or a public repository that gives closer information on how to reproduce the results obtained using the model, such as a link to a reproducible notebook or to a test suite
  \\
\verb|model_version|
 & optional
 & version string of the model, if available
 \\
\verb|environment|
 & required
 & free-text description of the software and hardware environment used to run FROG analysis tools, such as the versions of operating system, the hardware platform and the software tools; if possible, this includes a pointer to a machine-readable information that reproduces the environment, such as a versioned docker image name, or a version-lock file for a package management system
 \\
\verb|software|$\to$\verb|frog|
 & required
 & sub-object with a description of the FROG report generator (see below)
 \\
\verb|software|$\to$\verb|solver|
 & required
 & sub-object with a description of the solver software
 \\\addlinespace
\verb|software|$\to\ast\to$\verb|name|
 & required
 & name of the given software
 \\
\verb|software|$\to\ast\to$\verb|version|
 & required
 & version of the software
 \\
\verb|software|$\to\ast\to$\verb|url|
 & optional
 & URL from where the software may be obtained
 \\
\verb|software|$\to\ast\to$\verb|configuration|
 & optional
 & sub-object with a free-form description of the used software configuration (such as solver options or command line parameters)
 \\
\bottomrule
\end{tabular}
\caption[Contents of the FROG metadata file.]{Contents of the FROG metadata file. Names that contain an arrow ($\to$) represent properties of nested objects; asterisk ($\ast$) stands for any sub-object.}
\label{tab:metadata}
\end{table}

\subsection{Objective value report file}
\label{sec:obj}

For each objective that is described by the model file, the objective value report contains one entry row that describes the optimal objective value found by the flux balance analysis algorithm (\cref{sec:algos}) for each objective in the model. The required columns and the format and semantics of the entries are specified in~\cref{tab:objfields}.

\begin{table}\tablefont
\begin{tabular}{llp{30em}}
\toprule
Column & Format & Value \\
\midrule
\verb|model|
 & text
 & base name of the file, matching \verb|model.filename| in the metadata file
 \\
\verb|objective|
 & text
 & identifier of the objective
 \\
\verb|status|
 & solver status
 & termination status of the flux balance analysis solver run
 \\
\verb|value|
 & numeric
 & optimal objective value, missing if the problem is infeasible.
 \\
\bottomrule
\end{tabular}
\caption{Columns present in FROG objective value report file.}
\label{tab:objfields}
\end{table}

\subsection{Flux variability report file}
\label{sec:fva}
For each objective described by the model file and for each reaction, the flux variability report contains an entry that describes the minimal, maximal, and reference flux value through the given reaction at the optimal flux objective. The required columns and the format and semantics of the entries are specified in~\cref{tab:fvafields}.
A flux variability analysis should generate minima and maxima values (see \cref{sec:algos}) and may need to be included if the given subproblem is infeasible. The reference flux must be a valid optimal solution generated by a flux balance analysis; its value is only informative because the solution is typically not unique.

In case the base FBA problem for a given objective is infeasible, i.e., the objective is marked as infeasible in the objective report file (see~\cref{sec:obj}), no entries for that objective must be added.

\todo{rename flux to reference\_flux?}

\begin{table}\tablefont
\begin{tabular}{llp{30em}}
\toprule
Column & Format & Value \\
\midrule
\verb|model|
 & text
 & base name of the file, matching \verb|model.filename| in the metadata file
 \\
\verb|objective|
 & text
 & identifier of the objective that is optimized
 \\
\verb|reaction|
 & text
 & identifier of the reaction
 \\
\verb|flux|
 & numeric
 & the reference flux through the reaction in some computed optimal solution
 \\
\verb|status|
 & solver status
 & termination status of the flux variability analysis solver runs for the reaction, marked optimal if at least one of the subproblems yield a solution (see note in \cref{sec:fvastatus})
 \\
\verb|minimum|
 & numeric
 & minimal flux through the reaction with optimal objective, missing if the problem is infeasible
 \\
\verb|maximum|
 & numeric
 & maximal flux through the reaction with optimal objective, missing if the problem is infeasible
 \\
\verb|fraction_optimum|
 & numeric
 & fragment of objective value that must be retained by flux variability analysis (must be \verb|1.0|)
 \\
\bottomrule
\end{tabular}
\caption{Columns present in FROG flux variability report file.}
\label{tab:fvafields}
\end{table}

\subsubsection{Compound termination status of reaction minimization and maximization}
\label{sec:fvastatus}

Reaction variability report (\cref{tab:fvafields}) specifies that the status of minimization and maximization of the flux through the given reaction should be reported as a single solver status.
While this is a theoretically valid assumption, there are cases where solvers fail due to numeric problems and only produce the optimal solution in one direction.

To provide deterministically comparable reports in this case, this specification mandates that the solver status must be reported as optimal in case any optimal solution exists for any of the subproblems in the optimization.
As such, reporting optimality may be interpreted as merely stating that the software has found a proof that optimal solutions of the model can be found if the given variable becomes the optimization objective.
Infeasibility of individual subproblems as reported by the solver must still be reported properly by marking the respective maxima and minima as missing.

\subsection{Gene deletion report file}
\label{sec:genes}

For each objective described by the model file and for each reaction, the gene deletion report contains an entry describing the achievable objective value if the given gene is deleted (i.e., `knocked out'). The required columns and the format and semantics of the entries are specified in~\cref{tab:genefields}.

The gene deletion algorithm must be compatible with the single gene knockout procedure described in~\cref{sec:algos}.

\begin{table}\tablefont
\begin{tabular}{llp{30em}}
\toprule
Column & Format & Value \\
\midrule
\verb|model|
 & text
 & base name of the file, matching \verb|model.filename| in the metadata file
 \\
\verb|objective|
 & text
 & identifier of the objective that is optimized
 \\
\verb|gene|
 & text
 & identifier of the gene to be knocked out
 \\
\verb|status|
 & solver status
 & termination status of the corresponding solver run
 \\
\verb|value|
 & numeric
 & optimal objective value when the gene is knocked out, missing if the problem is infeasible
 \\
\bottomrule
\end{tabular}
\caption{Columns present in FROG gene deletion report file.}
\label{tab:genefields}
\end{table}

\subsection{Reaction deletion report file}
\label{sec:rxns}

For each objective described by the model file and for each reaction, the reaction deletion report contains an entry describing the achievable objective value if the given reaction is deleted from the model, as specified in \cref{sec:algos}. The required columns and the format and semantics of the entries are specified in~\cref{tab:rxnsfields}.

\begin{table}\tablefont
\begin{tabular}{llp{30em}}
\toprule
Column & Format & Value \\
\midrule
\verb|model|
 & text
 & base name of the file, matching \verb|model.filename| in the metadata file
 \\
\verb|objective|
 & text
 & identifier of the objective that is optimized
 \\
\verb|reaction|
 & text
 & identifier of the reaction to be deleted
 \\
\verb|status|
 & solver status
 & termination status of the corresponding solver run
 \\
\verb|value|
 & numeric
 & optimal objective value when the reaction is deleted from the model, missing if the problem is infeasible
 \\
\bottomrule
\end{tabular}
\caption{Columns present in FROG reaction deletion report file.}
\label{tab:rxnsfields}
\end{table}
