\chapter{Background and rationale}

% This is based on the curation/FBC biomodels site. We might want to keep these in sync.:
Constraint-based models are used to investigate metabolism in diverse conditions; particularly genome-scale metabolic models (GEMs) with thousands of reactions provide opportunities to analyze organism-specific metabolism. Community standards for consistent model reconstruction, curation, and sharing are crucial to ensure reproducibility, reliability, and FAIR sharing of the models. MEMOTE, a community tool was developed for standardized quality assessment of the models.

It is complicated to assess whether constraint-based models, including GEMs, are reproducible, because these models often have multiple solutions and the numerical values are not always enumerated in the published manuscripts. FROG is a community effort for standardized assessment of model reproducibility.

Reproducibility of results is the cornerstone of science and its assessment is an essential part of the curation of a model in the repository such as BioModels. Following discussions at dedicated breakout sessions at HARMONY2020\footnote{\url{https://docs.google.com/document/d/135Us6bOWaK4vr1ZClAXwWOjmmykEEwx_geSR8wJEmEY/edit?usp=sharing}}, COMBINE2020\footnote{\url{https://docs.google.com/document/d/1yGI5iAq_lTm1BNCb6GM0o0zlMNPRc6yOJv0iZHGTvOA/edit?usp=sharing}}, and HARMONY2021\footnote{\url{https://docs.google.com/document/d/1j62z9loPasxZ32t9S9eR6stO9ZHkBq9kDvhxG98lnmA/edit?usp=sharing}}, FROG analysis was developed as an ensemble of analyses of constraint-based models that generate standardized, numerically reproducible reference dataset for a given model, called a \emph{FROG report}. A corresponding collection of tools that generates FROG reports in a standardized schema was implemented. We propose that the modelers share FROG reports along with their model, which can be used by the modeler or curator to independently assess the reproducibility of a model, and ensure that the model data users can validate the results of their model interpretation.

FROG analysis is currently used in BioModels' workflow for the curation of constraint-based models along with the MEMOTE test suite~\cite{lieven2020memote}, which provides complementary functionality for validating the model annotation quality.

To allow retrospective curation of previously published models, authors may also submit a miniFROG report in addition to the autogenerated FROG report. The miniFROG report is a manually created data table in a standardized schema that lists some result values described in the manuscript, and links them to the values present in the autogenerated FROG report.

Sharing FROG (and possibly miniFROG) reports along with the model (typically in the SBML-FBC format~\cite{olivier2018sbmlfbc}) should enable assessment of reproducibility and curation of models, and thereby greatly enhance the reuse, extension and integration of constraint-based models in workflows for new knowledge generation. 

\section{Terms and definitions}

Requirement level of the wording used in technical sections of this specification (\cref{chap:spec,chap:test}), particularly terms \emph{must}, \emph{must not}, \emph{should}, \emph{should not}, and \emph{may}, is used as specified by RFC~2119~\cite{rfc2119}.

For the purposes of FROG, \emph{model} means exclusively a metabolic model encoded in the SBML format~\cite{hucka2003systems,keating2020sbml} with FBC extension~\cite{olivier2018sbmlfbc}.

\todo{specify `model user'}

\subsection{Metabolic model contents and algorithms}
\label{sec:algos}

Following the SBML-FBC specification, the model must uniquely specify a matrix $S$ (stoichiometry matrix), vectors $u$ (upper bounds), $l$ (lower bounds), and a finite sequence of vectors $c_i$ (an objective vector for each $i$) that define a valid set of optimization problems:
\begin{align*}
\max_x\ c_i^\intercal &\cdot x \\
\text{such that}\quad
x & \geq l \\
x & \leq u \\
S\cdot x &= 0
\end{align*}

The model must specify a sequence of identifiers $g$ where $g_i$ is the identifier of $i$-th gene, and $r$ where $r_i$ is the identifier of the $i$-th reaction and $|r| = |u| = |l|$.

Additionally, for each reaction the model must specify a Boolean formula $G_i$ (for $i$-th reaction) over variables labeled as $g_i$ (for $i$-th gene), giving the reaction-gene associations. The reaction-gene association formulas must possess an efficient evaluation method for a given assignment of variables, and should be preferably represented in disjunctive normal form (DNF, defined as $\bigvee_i\bigwedge_j o_{i,j}(g_j)$ where $(\forall i,j)\ o_{i,j}(x) \in \{x, \neg x, \top\}$).

\emph{Flux balance analysis (FBA) solution} for a given objective $i$ is any of the vectors $x$ in the above optimization problem that maximize the \emph{objective value} $c_i^\intercal \cdot x$.

\emph{Flux variability analysis (FVA) solution} is a pair of vectors of objective values generated by substituing the objective vector in the above optimization problem for $\pm e_i$ ($e_i$ is a vector that contains only zeros, except for a single 1 at position $i$) for each of $i$ reactions, and adding a constraint $c_i^\intercal\cdot x \geq a$ where $a$ is the maximized objective value given by a FBA of the original problem. Solutions for $-e_i$ are called flux minima through $i$-th reaction; correspondingly the solutions for $+e_i$ are called maxima.

\emph{Deletion of $i$-th reaction} in the model is carried out by setting $l_i = u_i = 0$.

\emph{Deletion of $i$-th gene} in the model is carried out by evaluating formulas $G$ for all reactions over a variable assignment $(\forall j)\ g_j = (j \neq i)$, and deleting all reactions for which their formula evaluates to $\bot$.~\cite{palsson2015systems}

\todo{maybe add examples of the math?}

\section{Aims and scope}

This document describes and specifies the FROG analysis, which is a set of methods to generate and evaluate constraint-based metabolic models. The main technical purpose of the FROG analysis is to assess the numerical reproducibility of model solution. Evaluating or comparing FROG reports generated by the model user to these generated by the model author scrutinizes the functionality of the software that generated the report, giving the following assurances:
\begin{itemize}
\item The software of the model user interprets a substantial portion of the model data in the same way as the software of the model author, thus expectably giving the same flux solutions (e.g., metabolic predictions) for a wide range of analyses,
\item the solver software computes a valid solution that is compatible with the expectations of the model authors.
\end{itemize}

The information stored in the FROG report may be used as a proof of reproducibility of the analyses presented in a manuscript. Indirectly, FROG report comparison may also be used to assess the compatibility and quality of modeling software.

In the current version, FROG analysis does \emph{not} aim to ensure the semantics and interpretability of model results or annotations; tests relevant for that purpose are supplied e.g.~by the MEMOTE suite~\cite{lieven2020memote}.

\subsection{Specification development and versioning}

The version string of this specification is \texttt{\frogspecversion}; the version string is calculated for each version of the specification from the git commit IDs and git tags. The versions assigned by tags must follow the semantic versioning specification~\cite{semver}.

The git repository that is used to develop, publish and archive the specification versions is hosted on GitHub at \url{\frogrepourl}.\todo{this should change to a less personal namespace} Development, changes, amendments and extensions of the specification should be preferably discussed via channels associated with the repository (issue tracker, pull requests) to enable and sustain the community involvement.

\section{Intended workflows}

This section describes the intended usage and handling of the FROG report files and associated software from the perspective of several workflows.

\begin{description}
\item[Model author workflow] Upon publishing the model, use FROG software to generate a FROG report. If applicable, verify that the values in the FROG report match the expectations (e.g., model solves correctly). Deposit the FROG report together with the model. Optionally, describe the correspondence of values in an associated manuscript to the values in the FROG report in a miniFROG report, and deposit the miniFROG file.
\item[Model curator workflow] Before publishing the model, generate several FROG reports using several software tools and solvers. Check that all generated reports are compatible, ensuring that the model does not pose interpretation ambiguity and that it is numerically robust. If supplied, also compare the reports to the FROG report submitted by the author.
\item[Model user workflow] Before using the model using a given analysis software and solver, generate a FROG report using a software tool that uses the same solver and possibly utilizes the same analysis software. Compare the report to the author's published FROG report, verifying that the analysis software interprets the model correctly and the basic results are reproducible.
\end{description}

\iffalse
\section{DEMO SECTION (copypaste stuff from here)}

Wording follows the RFC~2119~\cite{rfc2119}.

Also cite biomodels by~\citeauthor{malik2020biomodels}.

Refer to \cref{lst:example} and \cref{fig:example}.

\begin{listing}
\begin{lstlisting}
tets test
this that these
\end{lstlisting}
\caption[Example]{Example listing.}
\label{lst:example}
\end{listing}

\begin{figure}
\begin{center}\fbox{\Huge Figure!}\end{center}
\caption[Example]{Example figure.}
\label{fig:example}
\end{figure}

\begin{table}\tablefont
\begin{tabular}{lll}
\toprule
Column & And & Another \\
\midrule
a & b & c \\
d & e & f \\
\bottomrule
\end{tabular}
\caption[Example]{Example table.}
\label{tab:example}
\end{table}
\fi
