\chapter{Testing FROG report compatibility}
\label{chap:test}

Decision on whether the properties of a model covered by the FROG report are reproducible can be checked by comparing the FROG report associated with the published model (or otherwise given by the author as a warrant of model reproducibility) to the FROG report generated by the model user's software.
The primary outcome of the comparison is the assurance of the correctness of the model interpretation by user's software and of the computed numerical results, as discussed in \cref{sec:aims}.

While both reports, in comparison, would be ideally identical to the isomorphism of data representation, this is rarely the case because of the common differences in the diverse computational methods used to evaluate the model. At the same time, FROG software may choose not to generate non-mandatory parts of the models, and comparison methodology and results may depend on the presence of the non-mandatory parts.

\section{Comparison methods}

Individual values in the models are treated differently depending on the data type and the possibility of omitted values.
All individual values are assigned one of the following \emph{importance categories} that specify the comparison semantics:
\begin{itemize}
\item The \textbf{reproducible value} describes core result that should be obtainable from the model, such as the expected objective value.
\item The \textbf{consistent value} may differ in specific situations, such as based on the user's choice of parameters or model interpretation but should be consistent under default conditions, and the user should be informed in case of inconsistencies. Typical values that should be consistent among FROG reports are the possibly non-unique reference solutions and file contents hashes.
\item The \textbf{informative value} is a value that is generally variable and may differ between FROG runs and is useful only as a source of information and metadata for purposes not directly related to model analysis, such as error debugging and metadata collection. Typical informative values are the URLs and versions of models and software tools.
\end{itemize}

By default, the individual values are compared by simple equality (the same strings are the same value, and different strings are considered a comparison failure).
Unless specified otherwise, numeric values are treated as inexact (see \cref{sec:inexact}).
Several exceptional cases given by the specification (such as the reference solutions) are treated differently (see \cref{sec:cmpdata}).

The output of comparison is a set of messages of various severity (errors, warnings, information, optional debug information).
The comparison process should report any differences between the compared reports following the scheme in \cref{tab:importance}.
This scheme gives the model's user sufficient information to efficiently evaluate the possible problems that may occur while reproducing the model author's results.

The software should summarize the comparison outcome based on the maximum message severity. If no errors or warnings were generated, the comparison outcome should be marked as `FROG reproducible.' If warnings were generated but no errors, the outcome should be `FROG reproducible with warnings.'
If errors were generated, the outcome should be marked as `Not reproducible.'

\begin{table}\tablefont
\begin{tabular}{rcccc}
\toprule
 & Required values & \multicolumn{3}{c}{Optional values} \\ \cmidrule(rl){3-5}
Importance & & Both values present & One value missing & No values present \\
\midrule
Reproducible & error   & error   & warning & debug      \\
Consistent   & warning & warning & info    & no message \\
Informative  & info    & info    & debug   & no message \\
\bottomrule
\end{tabular}
\caption[Comparison failure reporting levels]{Reporting severities for value comparison failures for each importance category.}
\label{tab:importance}
\end{table}

\subsection{Treating numeric values}
\label{sec:inexact}

If not marked otherwise, FROG software must treat the numeric values as inexact and compare them only using approximate equality.
\begin{itemize}
\item The values are considered approximately equal if their absolute difference is within the \emph{absolute tolerance} ($T_\text{abs}$),
\item the values are considered approximately equal if their relative difference is within the \emph{relative tolerance} ($T_\text{rel}$),
\item otherwise, the values are considered non-equal.
\end{itemize}

For values $a, b$, the approximate equality may be mathematically evaluated as follows:
$$|a-b|<T_{\text{abs}} + T_{\text{rel}}\cdot \max\left\{|a|, |b|\right\}$$

FROG software should use default tolerance values close to $T_\text{abs}=10^{-6}$, $T_\text{rel}=10^{-4}$.

\section{Metadata comparison}

The values stored in JSON object fields are compared in pairs that share the same JSON object key (or key sequence in the case of nested objects, sometimes called ``property path'').
The comparison semantics is given by importance value for the keys as listed in \cref{tab:cmpmeta}.

In order to produce better reports, the software may use a specialized version-compatibility check that follows the semantics of SemVer~\cite{semver} instead of the exact string check for comparing the contents of fields \verb|frog_version| and \verb|model_version|.

\begin{table}\tablefont
\begin{tabular}{llp{30em}}
\toprule
Value & Importance & Failure semantics \\
\midrule
\verb|frog_version|
 & consistent
 & generated report semantics may differ
 \\
\verb|frog_id|
 & informative
 & identifier of the FROG report that is unique within a given data repository
 \\
\verb|model_filename|
 & consistent
 & compared models may differ
 \\
\verb|model_md5|
 & consistent
 & compared model data may differ
 \\
\verb|model_sha256|
 & consistent
 & compared model data may differ
 \\
\verb|model_file_url|
 & informative
 & used for resolving problems with base data
 \\
\verb|model_development_url|
 & informative
 & used for resolving reproducibility failures
 \\
\verb|model_version|
 & consistent
 & compared models likely differ
 \\
\verb|environment|
 & informative
 & used for reproducing the original analysis environment
 \\
\verb|software|$\to$ \dots
 & informative
 & useful for debugging and possibly reproducing the original model interpretation
 \\
\bottomrule
\end{tabular}
\caption[Comparison semantics of the FROG metadata file.]{Comparison semantics of the data in the FROG metadata file. Names that contain an arrow with dots ($\to$\dots) represent properties of any nested objects.}
\label{tab:cmpmeta}
\end{table}

\section{Tabular report data comparison}
\label{sec:cmpdata}

Entries in the tabular report files are compared in pairs of equal keys from both files, as listed in \cref{tab:cmpvals}.
The entry field \texttt{model} is not compared across the reports but must be equal to the \texttt{model.filename} value in the metadata file of the corresponding report.

Other fields are compared using the plain string or inexact numeric comparison semantic, following \cref{tab:cmpvals}.
As the only exception, reference flux solutions must be compared to the range given by files, as described in \cref{sec:cmpreference}.
For comparison, all numeric fields are considered optional; other fields are required.

\begin{table}\tablefont
\begin{tabular}{lllll}
\toprule
File & Comparison key & Column & Comparison & Importance \\
\midrule
Objective report & \texttt{objective}
   & \texttt{status} & string & reproducible \\
 & & \texttt{value} & inexact & reproducible \\\addlinespace
Flux variability report & \texttt{objective}, \texttt{reaction}
   & \texttt{flux} & \multicolumn{2}{l}{special (see \cref{sec:cmpreference})} \\
 & & \texttt{status} & string & reproducible \\
 & & \texttt{minimum} & inexact & reproducible \\
 & & \texttt{maximum} & inexact & reproducible \\
 & & \texttt{fraction\_optimum} & exact & reproducible \\\addlinespace
Gene deletion report & \texttt{objective}, \texttt{gene}
   & \texttt{status} & string & reproducible \\
 & & \texttt{value} & inexact & reproducible \\\addlinespace
Reaction deletion report & \texttt{objective}, \texttt{reaction}
   & \texttt{status} & string & reproducible \\
 & & \texttt{value} & inexact & reproducible \\
\bottomrule
\end{tabular}
\caption{Comparison methods and failure semantics for values in tabular report files.}
\label{tab:cmpvals}
\end{table}

\subsection{Comparing reference flux solutions}
\label{sec:cmpreference}

Testing exact equality of the reference flux solutions in column \texttt{flux} of the flux variability reports is irrelevant, as different analysis software and solvers may reach different, equally valid reference solutions.
Nevertheless, reporting and comparing the fluxes may give necessary evidence for model reproducibility (or irreproducibility).
The comparison in FROG is, thus, performed in the following steps:
\begin{itemize}
\item The reference flux value in one compared report is between the flux variability minimum and maximum, or is approximately equal to either of the extrema (as described in \cref{sec:inexact}) in the other compared report.
If not, an error message must be generated. This check is run symmetrically for both reports.

\item The reference flux in both compared reports is inside the flux variability range in the same report, using the same comparison method as in the previous case. If not, an error is generated. (This serves as a consistency check.)

\item If both reference fluxes are in range, the value is further treated as an informative numeric value.
\end{itemize}
