\chapter{Testing FROG report compatibility}
\label{chap:test}

Decision on whether the properties of a model covered by the FROG report are reproducible can be checked by comparing the FROG report associated with the published model (or otherwise given by the author as a warrant of model reproducibility) to the FROG report generated by the model user's software. The main outcome of the comparison is assurance of correctness of the model interpretation by user's software and of the computed numerical results, as discussed in \cref{sec:aims}.

While both reports in comparison would be ideally identical up to isomorphism of data representation, this is rarely the case because of the common differences in the diverse computational methods used to evaluate the model. At the same time, FROG software may choose not to generate non-mandatory parts of the models, and comparison methodology and result may depend on the presence of the non-mandatory parts.

\section{Comparison methods}

Individual values present in the models are treated differently depending on the data type and the possibility of the values being omitted. All individual values are assigned one of the following \emph{importance categories} that specify the comparison outcome:
\begin{itemize}
\item \textbf{Reproducible value} is a value that describes core result that should be obtainable from the model, such as the expected objective value.
\item \textbf{Consistent value} is a value that may differ in specific situations such as based on user choice of parameters or model interpretation, but should be consistent under default condition and user should be informed in case of inconsistencies. Typical values that should be consistent among FROG reports are the possibly non-unique reference solutions and file contents hashes.
\item \textbf{Informative value} is a value that is generally variable and may differ between FROG runs, and is useful only as a source of information and metadata for purposes not directly related to model analysis such as error debugging and metadata collection. Typical informative values are the URLs and versions of models and software tools.
\end{itemize}

\subsection{Treating omitted values}

\begin{table}\tablefont
\begin{tabular}{rccc}
\toprule
Importance & Both values present & One value missing & No values present \\
\midrule
Reproducible & error & warning & debug \\
Consistent & warning & info & no message \\
Informative & info & debug & no message \\
\bottomrule
\end{tabular}
\caption[Comparison failure reporting levels]{Reporting levels for value comparison failures for each importance category.}
\label{tab:importance}
\end{table}

\subsection{Treating numeric values}

\section{Metadata comparison}

\section{Main report data comparison}
\label{sec:cmpdata}
