\chapter{Testing FROG report compatibility}
\label{chap:test}

Decision on whether the properties of a model covered by the FROG report are reproducible can be checked by comparing the FROG report associated with the published model (or otherwise given by the author as a warrant of model reproducibility) to the FROG report generated by the model user's software. The main outcome of the comparison is assurance of correctness of the model interpretation by user's software and of the computed numerical results, as discussed in \cref{sec:aims}.

While both reports in comparison would be ideally identical up to isomorphism of data representation, this is rarely the case because of the common differences in the diverse computational methods used to evaluate the model. At the same time, FROG software may choose not to generate non-mandatory parts of the models, and comparison methodology and result may depend on the presence of the non-mandatory parts.

\section{Comparison methods}

Individual values present in the models are treated differently depending on the data type and the possibility of the values being omitted. All individual values are assigned one of the following \emph{importance categories} that specify the comparison semantics:
\begin{itemize}
\item \textbf{Reproducible value} is a value that describes core result that should be obtainable from the model, such as the expected objective value.
\item \textbf{Consistent value} is a value that may differ in specific situations such as based on user choice of parameters or model interpretation, but should be consistent under default condition and user should be informed in case of inconsistencies. Typical values that should be consistent among FROG reports are the possibly non-unique reference solutions and file contents hashes.
\item \textbf{Informative value} is a value that is generally variable and may differ between FROG runs, and is useful only as a source of information and metadata for purposes not directly related to model analysis such as error debugging and metadata collection. Typical informative values are the URLs and versions of models and software tools.
\end{itemize}

By default, the individual values are compared by simple equality (the same strings are the same value, different strings are considered a comparison failure). Numeric values are treated as inexact (see \cref{sec:inexact}). Several exceptional cases given by the specification (such as the reference solutions) are treated differently (see \cref{sec:cmpdata}).

The output of comparison is a set of messages of various severity (errors, warnings, information, optional debug information). Comparison process should report any differences found between the compared reports following the scheme in \cref{tab:importance}. The information gives the model user a sufficient information to see and evaluate the possible problems that may occur while reproducing the model author's results.

The software must summarize the comparison outcome based on the maximum message severity. If no errors or warnings were generated, the comparison outcome is marked as `FROG reproducible'. If warnings were generated but no errors, the outcome is `FROG reproducible with warnings'. If any errors were generated, the outcome is `Not reproducible'.

\begin{table}\tablefont
\begin{tabular}{rcccc}
\toprule
 & Required values & \multicolumn{3}{c}{Optional values} \\ \cmidrule(rl){3-5}
Importance & & Both values present & One value missing & No values present \\
\midrule
Reproducible & error & error & warning & debug \\
Consistent & warning & warning & info & no message \\
Informative & info & info & debug & no message \\
\bottomrule
\end{tabular}
\caption[Comparison failure reporting levels]{Reporting severities for value comparison failures for each importance category.}
\label{tab:importance}
\end{table}

\subsection{Treating numeric values}
\label{sec:inexact}

If not marked otherwise, FROG software must treat the numeric values as inexact and compare them only using approximate equality.
\begin{itemize}
\item The values are considered approximately equal if their absolute difference is within the \emph{absolute tolerance} ($T_\text{abs}$),
\item the values are considered approximately equal if their relative difference is within the \emph{relative tolerance} ($T_\text{rel}$),
\item otherwise the values are considered non-equal.
\end{itemize}

For values $a, b$, the approximate equality may be evaluated mathematically as follows:
\[\left(|a-b|<T_{\text{abs}}\right) \lor \left(|a-b|<T_{\text{rel}}\cdot \max\left\{|a|, |b|\right\}\right) \]

FROG software should use default tolerance values that are close to $T_\text{abs}=10^{-6}$, $T_\text{rel}=10^{-4}$.

\section{Metadata comparison}

The values stored in JSON object fields are compared in pairs that share the equal JSON object key (or key sequence in case of nested objects, sometimes called ``property path''). The comparison semantics is given by importance value for the keys as listed in \cref{tab:cmpmeta}.

\begin{table}\tablefont
\begin{tabular}{llp{30em}}
\toprule
Value & Importance & Failure semantics \\
\midrule
\verb|model.filename|
 & consistent
 & compared models may differ
 \\
\verb|model.md5|
 & consistent
 & compared model data may differ
 \\
\verb|model.sha256|
 & consistent
 & compared model data may differ
 \\
\verb|model.url|
 & informative
 & used for resolving problems with base data
 \\
\verb|model.development.url|
 & informative
 & used for resolving reproducibility failures
 \\
\verb|model.version|
 & consistent
 & compared models likely differ
 \\
\verb|environment|
 & informative
 & used for reproducing the original analysis environment
 \\
\verb|software|$\to$ \dots
 & informative
 & used for reproducing the original model interpretation
 \\
\verb|solver|$\to$ \dots
 & informative
 & may explain numerical errors
 \\
\bottomrule
\end{tabular}
\caption[Comparison semantics of FROG metadata file.]{Comparison semantics of the data in FROG metadata file. Names that contain an arrow with dots ($\to$\dots) represent properties of any nested objects.}
\label{tab:cmpmeta}
\end{table}

\section{Tabular report data comparison}
\label{sec:cmpdata}

Entries in the tabular report files are compared in pairs of equal keys from both files, as listed in \cref{tab:cmpvals}. Entry field \texttt{model} is not compared across the reports, but must be equal to the \texttt{model.filename} value in the metadata file of the corresponding report.

Other fields are compared using the plain string or inexact numeric comparison semantic, following \cref{tab:cmpvals}. As the only exception, comparison of reference flux solutions must be compared to the range given by files, as described in \cref{sec:cmpreference}. For comparison, all numeric fields are considered to be optional; other fields are required.

\begin{table}\tablefont
\begin{tabular}{lllll}
\toprule
File & Comparison key & Column & Comparison & Importance \\
\midrule
Objective report & \texttt{objective}
   & \texttt{status} & string & reproducible \\
 & & \texttt{value} & inexact & reproducible \\\addlinespace
Flux variability report & \texttt{objective}, \texttt{reaction}
   & \texttt{flux} & \multicolumn{2}{l}{special (see \cref{sec:cmpreference})} \\
 & & \texttt{status} & string & reproducible \\
 & & \texttt{minimum} & inexact & reproducible \\
 & & \texttt{maximum} & inexact & reproducible \\
 & & \texttt{fraction\_optimum} & exact & reproducible \\\addlinespace
Gene deletion report & \texttt{objective}, \texttt{gene}
   & \texttt{status} & string & reproducible \\
 & & \texttt{value} & inexact & reproducible \\\addlinespace
Reaction deletion report & \texttt{objective}, \texttt{reaction}
   & \texttt{status} & string & reproducible \\
 & & \texttt{value} & inexact & reproducible \\
\bottomrule
\end{tabular}
\caption{Comparison methods and failure semantics for values in tabular report files.}
\label{tab:cmpvals}
\end{table}

\subsection{Comparing reference flux solutions}
\label{sec:cmpreference}

Testing equality of the reference flux solutions in column \texttt{flux} of the flux variability reports does not make sense, as different analysis software and solvers may reach different, equally valid reference solutions. Nevertheless, reporting and comparing the fluxes may give important evidence for model reproducibility (or irreproducibility). The comparison in FROG is thus performed in the following steps:
\begin{itemize}
\item The reference flux in one compared report is larger than flux variability minimum and smaller than flux variability maximum, or approximately equal to either of the extrema (as described in \cref{sec:inexact}) in the other compared report. If not, error message must be generated. This check is run symmetrically for both reports.
\item The reference flux in both compared reports is inside of the flux variability range in the same report, using the same comparison method as in the previous case. If not, an error is generated. (This serves as a consistency check.)
\item If both reference fluxes are in range, the value is further treated as informative numeric value.
\end{itemize}
